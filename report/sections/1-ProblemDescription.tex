\section{Problem Description}
This report examines two time series tracking the evolution of the US GDP and CPIAUCSL from 1948 to 2021. \\
The US Gross Domestic Product (GDP) represents the total market value of goods and services produced within the United States. The Consumer Price Index for All Urban Consumers (CPIAUCSL) reflects the average cost of a basket of goods and services purchased by urban consumers. Both metrics are reported quarterly and have been seasonally adjusted. In particular, we analyze their percentage changes over time. \\
Our project aims to achieve the following objectives:
\begin{itemize}
    \item Independently fit the two time series using AR, MA, ARMA, and GARCH models.
    \item Fit the two time series together using a VAR model.
    \item Compare the results obtained from different models by calculating DIC and WAIC.
\end{itemize}
All models were implemented using JAGS with the following specifications:
\begin{itemize}
    \item 3 Chains.
    \item Total of 10,000 Iterations.
    \item 1,000 Burn-in Iterations.
\end{itemize}
Furthermore, we analyzed trace plots and autocorrelation plots to detect any issues and compared our results with those obtained from public libraries and functions. Detailed results can be found in the Appendix.